\section{Within the path of an AGN jet}

\begin{frame}{From jet deceleration to particle acceleration}

		{\scriptsize
	\begin{columns}
		\begin{column}{0.5\textwidth}
			\centering
			\includegraphics[width=1.1\linewidth]{images/cena_jet.jpg}
			Nearby galaxy Centaurus A in X-ray (Kraft et al 2001, Goodger et al 2010)
				\begin{block}{Entrainment in FRIs}
				Protons or heavier elements (stellar winds, clouds, SN, ambient medium) \\
			    $\rightarrow$ inexorable mass-loading and deceleration
						of the jet (De Young 1986, Bowman et al. 1996)
			\end{block}

		\end{column}
		\begin{column}{0.5\textwidth}
			\centering
			\includegraphics[width=.9\linewidth]{images/jet_wind.png}
				\begin{exampleblock}{Shocks}
				\begin{itemize}
					\item Balance between 2 supersonic matching flows (Komissarov 1994, Hubbard \& Blackman 2006)
					\item Conversion of $E_{k}$ into $U$ \\
						$\rightarrow$ expansion of the shock surface  \\
					\item Acceleration of $e^{-}$, $p$ or heavier ions\\
						$\rightarrow$ possible emissions up to $\gamma$-rays and
						acceleration of UHECR?
				\end{itemize}
			\end{exampleblock}

		\end{column}
	\end{columns}
	}
\end{frame}

\begin{frame}{Mixing and deceleration in FRIs}
	\begin{columns}
		\begin{column}{.5\textwidth}
			\includegraphics[width=.8\linewidth]{images/perucho_2014.png}
				{\tiny Mass-loading by stellar winds (Perucho et al. 2014)}

			\includegraphics[width=.7\linewidth]{images/perucho_2017.png} \\
				{\tiny Interaction with the jet/ISM shear layer (Perucho et al. 2017)}
		\end{column}
		\begin{column}{.5\textwidth}

			\begin{exampleblock}{Jet/cloud interaction}
				{\tiny 	Araudo et al. 2010, 2013;
				Barkov et al. 2010, 2012; \\
				Wykes et al. 2013, 2015;
				de la Cita et al. 2016}
			\end{exampleblock}
				{\tiny
			\begin{block}{Longo et al. in prep.}
				\begin{itemize}
				\item Induced mixing with ISM gas suggested for the deceleration
					of FRIs (Perucho 2020)
				\item Drive stars through the boundaries separating the ISM and the jet
				\item Among the ways to dissipate the jet energy
			    \end{itemize}
			\end{block}
				}
				
			\includegraphics[width=\linewidth]{images/jbs3_130.png}
		\end{column}
	\end{columns}
\end{frame}

\begin{frame}{Jet/star interaction time scale with the distance from the jet base}
	\begin{columns}
		{\scriptsize
		\begin{column}{0.5\textwidth}
			\begin{alertblock}{Close to the AGN | $z<10\,{\rm pc}$}
				\begin{itemize}
					\item Presence of large amounts of gas \\
						$\rightarrow$ lots of stars (RG, MS, AGB..) 
					\item Interaction with the outflow short ($t_{\rm int}=2R_{\rm j}/v_{\rm orb}$, typically 
							$\approx {\rm kyr}$)
								but frequent (short orbital period) (Kurfürst et al. 2024)\\
						$\rightarrow$ small mass loss per interaction  \\
				\end{itemize}
			\end{alertblock}
			\begin{block}{Far from the AGN | $z>{\rm kpc}$}
				\begin{itemize}
					\item The jet has expanded \\
							$\rightarrow$ Jet/star interaction time scale increases
								(typ. $\approx{\rm Myr}$)
					\item Stellar population typ. $\approx 1 {\rm star/pc}^{3}$ \\
						$\Rightarrow$ If SRG, eventual explosion
				\item 70 SN/century and ~0.01\% of them inside the jet (Vieyro et al. 2019)
				\end{itemize}
			\end{block}
		\end{column}
		\begin{column}{0.5\textwidth}
			\centering
			\includegraphics[width=\linewidth]{images/jet_mobstacles.png}
			\bf{A SRG can explode within the jet flow (Bosch-Ramon 2023) \\
		If the jet ram pressure becomes dominant over the SN remnant \\
						$\Rightarrow$ eventual disruption	}

		\end{column}}	
	\end{columns}
\end{frame}
